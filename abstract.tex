%%%%%%%%%%%%%%%%%%%%%%%%%%%%%%%%%%%%%%%%%%%%%%%%%%%%%%%%%%%%%%%%%
\subsection*{Abstract}
%%%%%%%%%%%%%%%%%%%%%%%%%%%%%%%%%%%%%%%%%%%%%%%%%%%%%%%%%%%%%%%%%

Network testbeds are essential research tools that have been
responsible for valuable network measurements and major advances in 
distributed systems research.  However, no single testbed can satisfy 
the requirements of every research project, prompting continual efforts 
to develop new testbeds.
%
Unfortunately, the common
practice is to re-implement functionality anew for each new testbed.

This work introduces a set of ready-to-use software components and
interfaces called Tsumiki to help researchers to rapidly prototype
custom networked testbeds without substantial effort.
%
We derive Tsumiki's design using a set of component and interface
design principles and demonstrate that Tsumiki can be used to
implement new, diverse, and useful types of testbeds. We detail
two such testbeds: a testbed composed of Android phones, and a testbed
that uses Docker for sandboxing.
%
%% We used these principles to derive a set of components called Tsumiki.
%% Tsumiki has proven effective at reducing development time and
%% improving functionality reuse across testbeds, while supporting a
%% diverse set of testbed use cases.
%
%Furthermore our experience demonstrates that \sysname 
%helps researchers build new types of testbeds, ones that
%reduces development time for new 
%would otherwise take substantial effort to construct from scratch.
%
A user study demonstrates that students with no prior experience with
networked testbeds were able to use Tsumiki to create a testbed with
new functionality and run an experiment on this testbed in under an
hour.

Furthermore, Tsumiki has been used in production
in multiple testbeds, resulting in installations on tens of thousands of 
devices and use by thousands of researchers.

%% Testbed federation allows existing testbeds to share resources and
%% credentials, however, constructing new testbeds remains a
%% substantial task.

% build new networked testbeds without substantial effort.

%The use in these environments demonstrates that freederation 
%% This work proposes a new methodology called \emph{freederation} that reduces
%% testbed construction effort.   
%% Freederation dictates how to delininate and design components to
%% improve their ability to be freely reused, re-purposed, and replaced.
%% We used this concept in the construction of a set of components called 
%% \sysname.
%\sysname has shown its utility and flexibility  through its use in
%build \sysname, a set of ready-made testbed
%components that make it easy for researchers to develop and deploy
%customized testbeds. % to suit their needs with minimal effort.
% JAC: I feel this isn't directly relevant here...
%The \sysname components have well defined interfaces and may be used to
%construct testbeds with a variety of different architectures. 
%% \sysname components have reduced the development time of
%% four different testbeds,
%, including
%BISmark~\cite{sundaresan2011broadband, sundaresan2014bismark}, social
%compute cloud~\cite{chard2010social, caton2014social}, and the
%Sensibility Testbed~\cite{sensibility-paper}. 
%The use in these environments demonstrates that freederation 
%
%These t
%% including a testbed that spans home
%% wireless routers used for network measurement, a testbeds composed of Android
%% phones, a testbed that uses Raspberry PI devices for censorship detection, and
%% a testbed with resource sharing among social links.
%We detail how freederation enabled the use of different sets of the \sysname
%components to interoperate to build in these testbeds. 
%Furthermore our experience demonstrates that \sysname 
%helps researchers build new types of testbeds, ones that
%reduces development time for new 
%would otherwise take substantial effort to construct from scratch.


