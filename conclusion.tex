%%%%%%%%%%%%%%%%%%%%%%%%%%%%%%%%%%%%%%%%%%%%%%%%%
\section{Conclusion}\label{sec-conclusion}
%%%%%%%%%%%%%%%%%%%%%%%%%%%%%%%%%%%%%%%%%%%%%%%%%

\cappos{Junk text, kill it!!!}
Building and deploying a new testbed is labor-intensive and
time-consuming. New testbed prototypes will continue to be developed
as new technologies appear and limitations of existing testbeds become
apparent. This paper describes a set of principles that we have found
effective in decomposing a testbed into a set of components and interfaces.
The resulting design, Tsumiki, forms a set of
ready-to-use components and interfaces that can be reused across platforms and
environments to rapidly prototype diverse networked testbeds. Tsumiki is
open source and is available at:\\
\url{http://tsumikifreederation.github.io}

